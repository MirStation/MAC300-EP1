\documentclass[a4paper,11pt]{article} 
\usepackage{times}
\usepackage{listings}
\usepackage{amsmath}
\usepackage[top=10mm, bottom=15mm, left=20mm, right=20mm]{geometry}
\usepackage{multirow}
\usepackage{hhline}

%% Escrevendo em português
\usepackage[brazil]{babel}
\usepackage[utf8]{inputenc}

\linespread{1.1} 

\newcommand{\sepitem}{\vspace{0.1in}\item} 
\newcommand{\titulo}{\item \textbf}
\begin {document}
\lstset{language=C}
\small{
\title{
{\small 
Departamento de Ciência da Computação \hfill IME/USP}\\\vspace{0.1in}
MAC0300 - Métodos Numéricos para Álgebra Linear - 2015/S2
}
\vspace{-0.6in} 
\author{
  António Martins Miranda (Nº 7644342) \{\textit{amartmiranda@gmail.com}\} \\
  \and 
  António Rui Castro Júnior (Nº 5984327) \{\textit{antonio.castro@usp.br}\}
  \vspace{-0.6in} 
}
\date{EP1 - Resolução de Sistemas de Equações Lineares - Relatório}
\maketitle
}
\vspace {-0.3in}
\thispagestyle{empty}  

\setlength{\parindent}{5ex}

\section{Sistemas definidos positivos}

Para a primeira parte do EP implementamos, de acordo com a exigências do enunciado e na linguagem de programação C, as seguintes funções: 
\begin{itemize}
	\item {\bf int cholcol (int n, double A[][nmax])} - implementa a decomposição de Cholesky orientada a colunas;
	\item {\bf int cholrow (int n, double A[][nmax])} - implementa a decomposição de Cholesky orientada a linhas;	
	\item {\bf int forwcol (int n, double A[][nmax], double b[])} - resolve um sistema do tipo $Ax=b$, orientado a colunas e usando substituição para frente;
	\item {\bf int forwrow (int n, double A[][nmax], double b[])} - resolve um sistema do tipo $Ax=b$, orientado a linhas e usando substituição para frente;
	\item {\bf int backcol (int n, double A[][nmax], double b[], int trans)} - resolve sistemas do tipo $Ax=b$ ou $A^{T}x=b$, orientado a colunas e usando a substituição para trás;
	\item {\bf int backrow (int n, double A[][nmax], double b[], int trans)} - resolve sistemas do tipo $Ax=b$ ou $A^{T}x=b$, orientado a linhas e usando a substituição para trás.
\end{itemize}
As funções cujo nome terminam com {\it col} são orientadas a coluna e as que terminam com {\it row} são orientadas a linha. Para mais detalhes sobre as funções, consultar a primeira parte do enuncidado do ep1.

\subsection{Decomposição de Cholesky}

Pseudocódigo da função {\it cholcol}:
\begin{lstlisting}
for j=1 to n
	for k=1 to j-1
		for i=j to n /*conhecido por cmod(j,k)*/
			a_ij = a_ij - a_ik * a_jk	
		end
	end
	if a_jj <= 0 
		return -1
	end
	a_jj = sqrt(a_jj) /*raiz quadrada de a_jj*/
	for k=j+1 to n /*conhecido por cdiv(j)*/
		a_kj = a_kj / a_jj
	end
end
return 0
\end{lstlisting}

\clearpage
\  \\
Pseudocódigo da função {\it cholrow}:
\begin{lstlisting}
for i=1 to n
	for j=1 to i-1
		for k=1 to j-1
			a_ij = a_ij - a_ik * a_jk	
		end
		a_ij = aij / a_jj
	end
	for j=1 to i-1
		a_ii = a_ii - a_ij * a_ij
	end
	if a_ii <= 0 
		return -1
	end
	a_ii = sqrt(a_ii) /*raiz quadrada de a_ii*/
end
return 0
\end{lstlisting}

\subsection{Resolução de um sistema $Ax=b$ com substituição para frente}

Pseudocódigo da função {\it forwcol}:
\begin{lstlisting}
for j=1 to n
	if a_jj <= 0 
		return -1
	end
	b_j = b_j / a_jj
	for i=j+1 to n
		b_i = b_i - a_ij * b_j
	end
end
return 0
\end{lstlisting}
Pseudocódigo da função {\it forwrow}:
\begin{lstlisting}
for i=1 to n
	for j=1 to i-1
		b_i = b_i - a_ij * b_j
	end
	if a_ii <= 0 
		return -1
	end
	b_i = b_i / a_ii
end
return 0
\end{lstlisting}

\subsection{Resolução de um sistema $Ax=b$ (ou $A^{T}x=b$) com substituição para trás}

Pseudocódigo da função {\it backcol}:
\begin{lstlisting}
if trans == 1
	for i=n to 1 /*decremento*/
		for j=i+1 to n
			b_i=b_i-a_ji*b_j
		end
		if a_ii == 0
			return -1
		end
		b_i=b_i/a_ii
	end
else
	for j=n to 1 /*decremento*/
		if a_jj == 0
			return -1
		end
		b_j=b_j/a_jj
		for i=j-1 to 1 /*decremento*/
			b_i=b_i-a_ij*b_j
		end
	end
end
return 0
\end{lstlisting}
Pseudocódigo da função {\it backrow}:
\begin{lstlisting}
if trans == 1
	for j=n to 1 /*decremento*/
		if a_jj == 0
			return -1
		end
		b_j=b_j/a_jj
		for i=j-1 to 1 /*decremento*/
			b_i=b_i-a_ji*b_j
		end
	end
else
	for i=n to 1 /*decremento*/
		for j=i+1 to n
			b_i=b_i-a_ij*b_j
		end
		if a_ii == 0
			return -1
		end
		b_i=b_i/a_ii
	end
end
return 0
\end{lstlisting}

\subsection{Tempos de execução da Decomposição de Cholesky}

\begin{table}[htb]
	\label{tab:inters}
	\large
	\centering
	\begin{tabular}{|c|c|c|c||c|c|c|}
		\hline
		\multirow{3}{*}{PROBLEMA} & \multicolumn{6}{|c|}{DECOMPOSIÇÃO DE CHOLESKY} \\
		\cline{2-7}
		& \multicolumn{3}{|c||}{ORIENTADO A LINHA} & \multicolumn{3}{|c|}{ORIENTADO A COLUNA} \\
		\cline{2-7}
		& $A=GG^{T}$ & $Gy=b$ & $G^{T}x=y$ & $A=GG^{T}$ & $Gy=b$ & $G^{T}x=y$ \\
		\hline
		\hline
		1 &  &  &  &  &  & \\
		2 &  &  &  &  &  & \\
		3 &  &  &  &  &  & \\
		4 &  &  &  &  &  & \\
		5 &  &  &  &  &  & \\
		6 &  &  &  &  &  & \\
		7 &  &  &  &  &  & \\
		\hline
	\end{tabular}
\end{table}
\  \\
{\bf Comentários:} 

Como usamos a linguagem C para programar as funções e nela as matrizes são armazenadas em memória por linha (e em sequência), era esperado que as funções orientadas a linha fossem mais eficiêntes que as funções orientadas a coluna, como podemos verificar pelos resultados da tabela. 

\section{Sistemas Gerais}

De acordo com o enunciado e usando e também usando a linguagem de programação C, implementamos para essa segunda parte as seguintes funções: 
\begin{itemize}
	\item {\bf int lucol (int n, double A[][nmax], int p[])} - implementa a decomposição LU orientada a colunas;
	\item {\bf int lurow (int n, double A[][nmax], int p[])} - implementa a decomposição LU orientada a linhas;	
	\item {\bf int sscol (int n, double A[][nmax], int p[], double b[])} - resolve um sistema do tipo $LUx=Pb$, orientado a colunas;
	\item {\bf int ssrow (int n, double A[][nmax], int p[], double b[])} - resolve um sistema do tipo $LUx=Pb$, orientado a linhas.
\end{itemize}
Como na primeira parte, as funções cujo nome terminam com {\it col} são orientadas a coluna e as que terminam com {\it row} são orientadas a linha. Para mais detalhes sobre as funções, consultar a segunda parte do enuncidado do ep1.

\subsection{Decomposição LU}

Pseudocódigo da função {\it lucol}:
\begin{lstlisting}
for k=1 to n-1
	imax=k
	for i=k+1 to n
		if (|a_ik| > |a_imax,k|)
			imax=i
		end
		p(k)=imax /*p(k) vetor permutacao*/
	end
	if p(k)!=k
		for j=1 to n
			tmp=a_kj
			a_kj=a_p(k),j
			a_p(k),j=tmp
		end
	end
	if a_kk == 0
		return -1
	end
	for i=k+1 to n
		a_ik=a_ik / a_kk
	end
	for j=k+1 to n
		for i=k+1 to n
			a_ij=a_ij-a_kj*a_ik
		end
	end
	if a_nn == 0
		return -1
	end
end
return 0
\end{lstlisting}
Pseudocódigo da função {\it lurow}:
\begin{lstlisting}
for k=1 to n-1
	imax=k
	for i=k+1 to n
		if (|a_ik| > |a_imax,k|)
			imax=i
		end
		p(k)=imax /*p(k) vetor permutacao*/
	end
	if p(k)!=k
		for j=1 to n
			tmp=a_kj
			a_kj=a_p(k),j
			a_p(k),j=tmp
		end
	end
	if a_kk == 0
		return -1
	end
	for i=k+1 to n
		a_ik=a_ik / a_kk
		for j=k+1 to n
			a_ij=a_ij-a_kj*a_ik
		end
	end
	if a_nn == 0
		return -1
	end
end
return 0
\end{lstlisting}

\subsection{Resolução de um sistema $LUx=Pb$}

Pseudocódigo da função {\it sscol}:
\begin{lstlisting}
for i=1 to n-1
	tmp=b_i
	b_i=b_p(i) /*p(i) vetor permutacao*/
	b_p(i)=tmp
end
for j=1 to n
	for i=j+1 to n
		b_i=b_i-a_ij*b_j
	end
end
for j=n to 1 /*decremento*/
	if a_jj == 0
		return -1
	end
	b_j=b_j/a_jj
	for i=1 to j-1
		b_i=b_i-a_ij*b_j
	end
end
return 0
\end{lstlisting}
Pseudocódigo da função {\it ssrow}:
\begin{lstlisting}
for i=1 to n-1
	tmp=b_i
	b_i=b_p(i) /*p(i) vetor permutacao*/
	b_p(i)=tmp
end
for i=1 to n
	for j=1 to i-1
		b_i=b_i-a_ij*b_j
	end
end
for i=n to 1 /*decremento*/
	if a_ii == 0
		return -1
	end
	for j=i+1 to n
		b_i=b_i-a_ij*b_j
	end
	b_i=b_i/a_ii
end
return 0
\end{lstlisting}

\subsection{Tempos de execução da Decomposição LU}

\begin{table}[htb]
	\label{tab:inters}
	\large
	\centering
	\begin{tabular}{|c|c|c||c|c|}
		\hline
		\multirow{3}{*}{PROBLEMA} & \multicolumn{4}{|c|}{DECOMPOSIÇÃO LU} \\
		\cline{2-5}
		& \multicolumn{2}{|c||}{ORIENTADO A LINHA} & \multicolumn{2}{|c|}{ORIENTADO A COLUNA} \\
		\cline{2-5}
		& $PA=LU$ & $LUx=Pb$ & $PA=LU$ & $LUx=Pb$ \\
		\hline
		\hline
		1 &  &  &  &\\
		2 &  &  &  &\\
		3 &  &  &  &\\
		4 &  &  &  &\\
		5 &  &  &  &\\
		6 &  &  &  &\\
		7 &  &  &  &\\
		\hline
	\end{tabular}
\end{table}
\  \\
{\bf Comentários:}

Como podemos verificar, as funções orientadas a linha são mais eficiêntes que as funções orientadas a coluna, visto que C é uma linguagem que armazena matrizes na memória por linha (e em sequência). 

Pelo pseudocódigo da decomposição de Cholesky verificamos que a ela é constituída por uma operação de custo $O(\frac{n^{2}}{2})$ ({\it cdiv}) e outra de custo $O(n^{3})$ ({\it cmod}), enquanto que a decomposição LU é constituída por uma operação de custo $O(n^{2})$ (permutação da linha que contém o elemento máximo em módulo com a linha do pivot), uma operação de custo $O(\frac{n^{2}}{2})$ (escolha do elemento máximo em módulo na coluna do pivot e abiaxo dele) e uma operação de custo $O(n^{3})$ (zerar os elementos abaixo do pivot usando operações elementares e armazenar essas operações nas posições zeradas). Logo, é natural que a decomposição LU seja mais custosa e aproximadamente o dobro da decomposição de Cholesky($2 \times O(n^{3}+\frac{n^{2}}{2}) \approx O(n^{3}+n^{2}) \approx O(n^{3}+\frac{3n^{2}}{2}) $). 

\vfill

\raggedleft
{\sc Setembro/2015}

\end{document}
